\documentclass{beamer}
\usepackage[utf8]{inputenc}
\usepackage[T1]{fontenc}
\usetheme{Copenhagen}
\usecolortheme{beaver}

\title{MultiPath User Datagram Protocol}
\author{Nikola Hardi}
\institute{FTN - Fakultet Tehničkih Nauka - Novi Sad}

\begin{document}
    \begin{frame}
        \titlepage
    \end{frame}

\section{Opis projekta}
    \begin{frame}
        \frametitle{Opis zahteva, po prioritetima}

        \begin{enumerate}
            \item{Raspodela resursa - prenos putem svih pogodnih kanala}
            \item{Pouzdanost - automatska retransmisija (ARQ)}
            \item{Robusnost - promena lokalne ili udaljene konfiguracije}
            \item{Brzina prenosa - iskoristiti potencijal raspoloživih kanala}
        \end{enumerate}
    \end{frame}

    \begin{frame}
        \frametitle{Dodatni zahtevi}
        \begin{itemize}
            \item{Implementacija pogodna za dalje istraživanje}
            \item{Jednostavnost upotrebe sa strane korisnika (autokonfiguracija)}
            \item{Automatska konfiguracija i otkrivanje okruženja}
            \item{Dosledna arhitektura - slanje i prijem samo u jednoj tački}
            \item{Reč je o streamu proizvoljne i nepoznate dužine}
            \item{Full Duplex prenos, multimaster}
        \end{itemize}
    \end{frame}

    \begin{frame}
        \frametitle{Pretpostavke}

        \begin{itemize}
            \item{Proizvoljan broj kanala}
            \item{Pri inicijalizaciji su dostupni svi kanali}
            \item{Implementacija prilagođena uobičajenim bibliotekama.
                Blokirajuća funkcija za prijem paketa i jednostavna promena
                TCP/IP sloja i uklanjanje PCAP biblioteke}
            \item{POSIX/Linux okruženje}
        \end{itemize}
    \end{frame}
    \begin{frame}
        \frametitle{Pomoćne biblioteke:  *\_utils}
        \begin{itemize}
            \item[eth]{ethernet okviri}
            \item[ip]{IP datagrami}
            \item[udp]{UDP datagrami}
            \item[mpudp]{MPUDP paketi i korisničke funkcije}
            \item[pcap]{proširenja pcap biblioteke}
            \item[net]{manipulacija mrežnim adresama}
        \end{itemize}

        \begin{itemize}
            \item{Građenje paketa}
            \item{Manipulacija zaglavljima}
            \item{Postavljanje korisničkih podataka (payload)}
            \item{Deskriptori mrežnih interfejsa}
        \end{itemize}
    \end{frame}

\section{Opis protokola}
    \begin{frame}
        \frametitle{Tipovi paketa}

        \begin{itemize}
            \item{konfiguracija - opis konfiguracije kanala}
            \item{korisnicki podaci}
            \item{potvrda prijema}
        \end{itemize}
    \end{frame}

    \begin{frame}
        \frametitle{Uspostavljanje veze}

        \begin{itemize}
            \item{Kreiranje lokalne konfiguracije}
            \item{Objavljivanje lokalne konfiguracije}
            \item{Prihvatanje nove udaljene konfiguracije}
            \item{Uparivanje kanala i početak prenosa korisničkih podataka}
        \end{itemize}
    \end{frame}


    \begin{frame}
        \frametitle{Slanje podataka}

        \begin{itemize}
            \item{Upravljač kanala preuzima nove podatke za prenos}
            \item{Po slanju paketa, paketi sa podacima se smeštaju u
                    red za čekanje potvrde o prijemu.}
            \item{Po prijemu potvrde, odgovarajući paket se uklanja iz
                    reda za čekanje potvrde o prijemu.}
            \item{Kada je primljena potvrda za paket koji je zaista u
                    redu za čekanje potvrde o prijemu, tada se taj
                    paket uklanja i vrši se pomeranje prozora.}
            \item{Ako se pomeranjem prozora oslobodilo mesto,
                    prihvataju se i šalju novi korisnički podaci.}
            \item{Slanje potvrde o prijemu i konfiguracije ne zahteva potvrdu,
                    pa ni mesto u redu za čekanje potvrde o prijemu.}
        \end{itemize}
    \end{frame}

    \begin{frame}
        \frametitle{Prijem podataka}

        \begin{itemize}
            \item{Po inicijalizaciji počinje prijem na svakom
                    odgovarajućem kanalu.}
            \item{Ukoliko je primljen paket sa konfiguracijom novijom
                    od poslednje poznate udaljene konfiguracije,
                    tada se ažurira udaljena konfiguracija i ponovo se
                    uparuju kanali.}
            \item{Ukoliko je primljen paket sa podacima i ti podaci
                    mogu da budu prihvaćeni, oni se smeštaju u korisnički
                    red sa dolaznim paketima i šalje se potvrda o prijemu.}
            \item{Ukoliko je primljen paket potvrde, proverava se red za
                    čekanje potvrde i traži se odgovarajući paket. Prozor
                    za slanje se pomera ukoliko je moguće.}
        \end{itemize}
    \end{frame}

    \begin{frame}
        \frametitle{Retransmisija}

        \begin{itemize}
            \item{Pri slanju svakog korisničkog paketa beleži se vreme slanja.}
            \item{Periodično se proverava red za čekanje potvrde. Za svaki
                    paket za koji nije stigla potvrda, poredi se poslednje
                    vreme slanja sa trenutnim vremenom i ukoliko je razlika
                    veća od zadate vrši se retransmisija.}
            \item{Pri svakoj retransmisiji se beleži broj koliko puta je paket
                    poslat. Ukoliko je neki paket poslat više puta od zadate
                    granice, tada se kanal smatra loše konfigurisanim ili
                    isključenim.}
        \end{itemize}
    \end{frame}

    \begin{frame}
        \frametitle{Rekonfiguracija (uparivanje linkova)}

        \begin{itemize}
            \item{Jedna sesija se sastoji od jednog ili više uparenih kanala.}
            \item{Pri inicijalizaciji i tokom rada se prati stanje lokalne
                    konfiguracije. Kada dođe do promene lokalne konfiguracije,
                    započinje proces rekonfiguracije.}
            \item{Po prijemu nove udaljene konfiguracije, započinje proces
                    rekonfiguracije.}
            \item{Rekonfiguracija se sastoji od:}
                \begin{itemize}
                    \item{pražnjenja redova za čekanje potvrde iz svih
                        kanala u rastućem redosledu ID paketa.}
                    \item{uparivanja lokalnih i udaljenih kanala}
                \end{itemize}
            \end{itemize}
    \end{frame}

    \begin{frame}
        \frametitle{Lokalna manipulacija paketima}

        \begin{itemize}
            \item{Korisniku su dostupni samo dolazni i odlazni red}
            \item{Paketi mogu da dolaze u i dolaze u kanal samo u dolaznoj
                    odnosno odlaznoj niti.}
            \item{Svaki kanal ima svoje dolazne/odlazne niti.}
        \end{itemize}
    \end{frame}

\section{Primer konkretne implementacija}
    \begin{frame}
        \frametitle{Terminologija: Monitor}

        \begin{itemize}
            \item[monitor]{upravljanje jednom sesijom}
            \item[worker]{upravljanje jednim kanalom}
        \end{itemize}

        Monitor je zadužen za:
        \begin{itemize}
            \item{Komunikaciju sa korisnikom}
            \item{Inicijalizaciju i konfiguraciju sistema}
            \item{Praćenje lokalne konfiguracije}
            \item{Prihvatanje udaljene konfiguracije}
        \end{itemize}
    \end{frame}

    \begin{frame}
        \frametitle{Terminologija: Worker}

        Worker je zadužen za:
        \begin{itemize}
            \item{Prijem paketa i prosleđivanje na odgovarajuće mesto u sistemu}
            \item{Pokušaj reasembliranja poruke (smeštanje korisničkih podataka
                    u odgovarajućem redosledu)}
            \item{Praćenje stanja globalnog odlaznog reda i prihvatanje
                    novih korisničkih podataka ukoliko je to moguće}
            \item{Praćenje stanja reda za slanje konfiguracije}
            \item{Praćenje stanja pomoćnog reda za rekonfiguraciju}
            \item{Praćenje stanja reda za retransmisiju}
            \item{Slanje potvrde o prijemu}
            \item{Pomeranje lokalnog klizećeg prozora}
            \item{Pristup fizičkom sloju}
        \end{itemize}
    \end{frame}

    \subsection{Monitor}
    \begin{frame}
        \frametitle{Arhitetura sistema}
    \end{frame}

    \begin{frame}
        \frametitle{Broadcasting}
    \end{frame}

    \begin{frame}
        \frametitle{mpudp\_config\_t \& mpudp\_if\_desc\_t}
    \end{frame}

    \begin{frame}
        \frametitle{Rekonfiguracija}
    \end{frame}

    \begin{frame}
        \frametitle{Prenos korisničkih podataka}

    \end{frame}

    \subsection{Worker}
    \begin{frame}
        \frametitle{Tx Thread}
    \end{frame}

    \begin{frame}
        \frametitle{Rx Thread}
    \end{frame}

    \begin{frame}
        \frametitle{Tx Watcher}
    \end{frame}

    \begin{frame}
        \frametitle{ARQ Watcher}
    \end{frame}


\end{document}
